



    
% !TeX spellcheck = de_DE
% !TeX encoding = UTF-8
\documentclass{scrreprt}

    
    
    \usepackage[T1]{fontenc}
    % Nicer default font (+ math font) than Computer Modern for most use cases
    \usepackage{mathpazo}

    % Basic figure setup, for now with no caption control since it's done
    % automatically by Pandoc (which extracts ![](path) syntax from Markdown).
    \usepackage{graphicx}
    % We will generate all images so they have a width \maxwidth. This means
    % that they will get their normal width if they fit onto the page, but
    % are scaled down if they would overflow the margins.
    \makeatletter
    \def\maxwidth{\ifdim\Gin@nat@width>\linewidth\linewidth
    \else\Gin@nat@width\fi}
    \makeatother
    \let\Oldincludegraphics\includegraphics
    % Set max figure width to be 80% of text width, for now hardcoded.
    \renewcommand{\includegraphics}[1]{\Oldincludegraphics[width=.8\maxwidth]{#1}}
    % Ensure that by default, figures have no caption (until we provide a
    % proper Figure object with a Caption API and a way to capture that
    % in the conversion process - todo).
    \usepackage{caption}
    \DeclareCaptionLabelFormat{nolabel}{}
    \captionsetup{labelformat=nolabel}

    \usepackage{adjustbox} % Used to constrain images to a maximum size 
    \usepackage{xcolor} % Allow colors to be defined
    \usepackage{enumerate} % Needed for markdown enumerations to work
    \usepackage{geometry} % Used to adjust the document margins
    \usepackage{amsmath} % Equations
    \usepackage{amssymb} % Equations
    \usepackage{textcomp} % defines textquotesingle
    % Hack from http://tex.stackexchange.com/a/47451/13684:
    \AtBeginDocument{%
        \def\PYZsq{\textquotesingle}% Upright quotes in Pygmentized code
    }
    \usepackage{upquote} % Upright quotes for verbatim code
    \usepackage{eurosym} % defines \euro
    \usepackage[mathletters]{ucs} % Extended unicode (utf-8) support
    \usepackage[utf8x]{inputenc} % Allow utf-8 characters in the tex document
    \usepackage{fancyvrb} % verbatim replacement that allows latex
    \usepackage{grffile} % extends the file name processing of package graphics 
                         % to support a larger range 
    % The hyperref package gives us a pdf with properly built
    % internal navigation ('pdf bookmarks' for the table of contents,
    % internal cross-reference links, web links for URLs, etc.)
    \usepackage{hyperref}
    \usepackage{longtable} % longtable support required by pandoc >1.10
    \usepackage{booktabs}  % table support for pandoc > 1.12.2
    \usepackage[inline]{enumitem} % IRkernel/repr support (it uses the enumerate* environment)
    \usepackage[normalem]{ulem} % ulem is needed to support strikethroughs (\sout)
                                % normalem makes italics be italics, not underlines
    

    
    
    % Colors for the hyperref package
    \definecolor{urlcolor}{rgb}{0,.145,.698}
    \definecolor{linkcolor}{rgb}{.71,0.21,0.01}
    \definecolor{citecolor}{rgb}{.12,.54,.11}

    % ANSI colors
    \definecolor{ansi-black}{HTML}{3E424D}
    \definecolor{ansi-black-intense}{HTML}{282C36}
    \definecolor{ansi-red}{HTML}{E75C58}
    \definecolor{ansi-red-intense}{HTML}{B22B31}
    \definecolor{ansi-green}{HTML}{00A250}
    \definecolor{ansi-green-intense}{HTML}{007427}
    \definecolor{ansi-yellow}{HTML}{DDB62B}
    \definecolor{ansi-yellow-intense}{HTML}{B27D12}
    \definecolor{ansi-blue}{HTML}{208FFB}
    \definecolor{ansi-blue-intense}{HTML}{0065CA}
    \definecolor{ansi-magenta}{HTML}{D160C4}
    \definecolor{ansi-magenta-intense}{HTML}{A03196}
    \definecolor{ansi-cyan}{HTML}{60C6C8}
    \definecolor{ansi-cyan-intense}{HTML}{258F8F}
    \definecolor{ansi-white}{HTML}{C5C1B4}
    \definecolor{ansi-white-intense}{HTML}{A1A6B2}

    % commands and environments needed by pandoc snippets
    % extracted from the output of `pandoc -s`
    \providecommand{\tightlist}{%
      \setlength{\itemsep}{0pt}\setlength{\parskip}{0pt}}
    \DefineVerbatimEnvironment{Highlighting}{Verbatim}{commandchars=\\\{\}}
    % Add ',fontsize=\small' for more characters per line
    \newenvironment{Shaded}{}{}
    \newcommand{\KeywordTok}[1]{\textcolor[rgb]{0.00,0.44,0.13}{\textbf{{#1}}}}
    \newcommand{\DataTypeTok}[1]{\textcolor[rgb]{0.56,0.13,0.00}{{#1}}}
    \newcommand{\DecValTok}[1]{\textcolor[rgb]{0.25,0.63,0.44}{{#1}}}
    \newcommand{\BaseNTok}[1]{\textcolor[rgb]{0.25,0.63,0.44}{{#1}}}
    \newcommand{\FloatTok}[1]{\textcolor[rgb]{0.25,0.63,0.44}{{#1}}}
    \newcommand{\CharTok}[1]{\textcolor[rgb]{0.25,0.44,0.63}{{#1}}}
    \newcommand{\StringTok}[1]{\textcolor[rgb]{0.25,0.44,0.63}{{#1}}}
    \newcommand{\CommentTok}[1]{\textcolor[rgb]{0.38,0.63,0.69}{\textit{{#1}}}}
    \newcommand{\OtherTok}[1]{\textcolor[rgb]{0.00,0.44,0.13}{{#1}}}
    \newcommand{\AlertTok}[1]{\textcolor[rgb]{1.00,0.00,0.00}{\textbf{{#1}}}}
    \newcommand{\FunctionTok}[1]{\textcolor[rgb]{0.02,0.16,0.49}{{#1}}}
    \newcommand{\RegionMarkerTok}[1]{{#1}}
    \newcommand{\ErrorTok}[1]{\textcolor[rgb]{1.00,0.00,0.00}{\textbf{{#1}}}}
    \newcommand{\NormalTok}[1]{{#1}}
    
    % Additional commands for more recent versions of Pandoc
    \newcommand{\ConstantTok}[1]{\textcolor[rgb]{0.53,0.00,0.00}{{#1}}}
    \newcommand{\SpecialCharTok}[1]{\textcolor[rgb]{0.25,0.44,0.63}{{#1}}}
    \newcommand{\VerbatimStringTok}[1]{\textcolor[rgb]{0.25,0.44,0.63}{{#1}}}
    \newcommand{\SpecialStringTok}[1]{\textcolor[rgb]{0.73,0.40,0.53}{{#1}}}
    \newcommand{\ImportTok}[1]{{#1}}
    \newcommand{\DocumentationTok}[1]{\textcolor[rgb]{0.73,0.13,0.13}{\textit{{#1}}}}
    \newcommand{\AnnotationTok}[1]{\textcolor[rgb]{0.38,0.63,0.69}{\textbf{\textit{{#1}}}}}
    \newcommand{\CommentVarTok}[1]{\textcolor[rgb]{0.38,0.63,0.69}{\textbf{\textit{{#1}}}}}
    \newcommand{\VariableTok}[1]{\textcolor[rgb]{0.10,0.09,0.49}{{#1}}}
    \newcommand{\ControlFlowTok}[1]{\textcolor[rgb]{0.00,0.44,0.13}{\textbf{{#1}}}}
    \newcommand{\OperatorTok}[1]{\textcolor[rgb]{0.40,0.40,0.40}{{#1}}}
    \newcommand{\BuiltInTok}[1]{{#1}}
    \newcommand{\ExtensionTok}[1]{{#1}}
    \newcommand{\PreprocessorTok}[1]{\textcolor[rgb]{0.74,0.48,0.00}{{#1}}}
    \newcommand{\AttributeTok}[1]{\textcolor[rgb]{0.49,0.56,0.16}{{#1}}}
    \newcommand{\InformationTok}[1]{\textcolor[rgb]{0.38,0.63,0.69}{\textbf{\textit{{#1}}}}}
    \newcommand{\WarningTok}[1]{\textcolor[rgb]{0.38,0.63,0.69}{\textbf{\textit{{#1}}}}}
    
    
    % Define a nice break command that doesn't care if a line doesn't already
    % exist.
    \def\br{\hspace*{\fill} \\* }
    % Math Jax compatability definitions
    \def\gt{>}
    \def\lt{<}
    % Document parameters
    \title{03-Handout-Solution-Only}
    
    
    

    % Pygments definitions
    
\makeatletter
\def\PY@reset{\let\PY@it=\relax \let\PY@bf=\relax%
    \let\PY@ul=\relax \let\PY@tc=\relax%
    \let\PY@bc=\relax \let\PY@ff=\relax}
\def\PY@tok#1{\csname PY@tok@#1\endcsname}
\def\PY@toks#1+{\ifx\relax#1\empty\else%
    \PY@tok{#1}\expandafter\PY@toks\fi}
\def\PY@do#1{\PY@bc{\PY@tc{\PY@ul{%
    \PY@it{\PY@bf{\PY@ff{#1}}}}}}}
\def\PY#1#2{\PY@reset\PY@toks#1+\relax+\PY@do{#2}}

\expandafter\def\csname PY@tok@w\endcsname{\def\PY@tc##1{\textcolor[rgb]{0.73,0.73,0.73}{##1}}}
\expandafter\def\csname PY@tok@c\endcsname{\let\PY@it=\textit\def\PY@tc##1{\textcolor[rgb]{0.25,0.50,0.50}{##1}}}
\expandafter\def\csname PY@tok@cp\endcsname{\def\PY@tc##1{\textcolor[rgb]{0.74,0.48,0.00}{##1}}}
\expandafter\def\csname PY@tok@k\endcsname{\let\PY@bf=\textbf\def\PY@tc##1{\textcolor[rgb]{0.00,0.50,0.00}{##1}}}
\expandafter\def\csname PY@tok@kp\endcsname{\def\PY@tc##1{\textcolor[rgb]{0.00,0.50,0.00}{##1}}}
\expandafter\def\csname PY@tok@kt\endcsname{\def\PY@tc##1{\textcolor[rgb]{0.69,0.00,0.25}{##1}}}
\expandafter\def\csname PY@tok@o\endcsname{\def\PY@tc##1{\textcolor[rgb]{0.40,0.40,0.40}{##1}}}
\expandafter\def\csname PY@tok@ow\endcsname{\let\PY@bf=\textbf\def\PY@tc##1{\textcolor[rgb]{0.67,0.13,1.00}{##1}}}
\expandafter\def\csname PY@tok@nb\endcsname{\def\PY@tc##1{\textcolor[rgb]{0.00,0.50,0.00}{##1}}}
\expandafter\def\csname PY@tok@nf\endcsname{\def\PY@tc##1{\textcolor[rgb]{0.00,0.00,1.00}{##1}}}
\expandafter\def\csname PY@tok@nc\endcsname{\let\PY@bf=\textbf\def\PY@tc##1{\textcolor[rgb]{0.00,0.00,1.00}{##1}}}
\expandafter\def\csname PY@tok@nn\endcsname{\let\PY@bf=\textbf\def\PY@tc##1{\textcolor[rgb]{0.00,0.00,1.00}{##1}}}
\expandafter\def\csname PY@tok@ne\endcsname{\let\PY@bf=\textbf\def\PY@tc##1{\textcolor[rgb]{0.82,0.25,0.23}{##1}}}
\expandafter\def\csname PY@tok@nv\endcsname{\def\PY@tc##1{\textcolor[rgb]{0.10,0.09,0.49}{##1}}}
\expandafter\def\csname PY@tok@no\endcsname{\def\PY@tc##1{\textcolor[rgb]{0.53,0.00,0.00}{##1}}}
\expandafter\def\csname PY@tok@nl\endcsname{\def\PY@tc##1{\textcolor[rgb]{0.63,0.63,0.00}{##1}}}
\expandafter\def\csname PY@tok@ni\endcsname{\let\PY@bf=\textbf\def\PY@tc##1{\textcolor[rgb]{0.60,0.60,0.60}{##1}}}
\expandafter\def\csname PY@tok@na\endcsname{\def\PY@tc##1{\textcolor[rgb]{0.49,0.56,0.16}{##1}}}
\expandafter\def\csname PY@tok@nt\endcsname{\let\PY@bf=\textbf\def\PY@tc##1{\textcolor[rgb]{0.00,0.50,0.00}{##1}}}
\expandafter\def\csname PY@tok@nd\endcsname{\def\PY@tc##1{\textcolor[rgb]{0.67,0.13,1.00}{##1}}}
\expandafter\def\csname PY@tok@s\endcsname{\def\PY@tc##1{\textcolor[rgb]{0.73,0.13,0.13}{##1}}}
\expandafter\def\csname PY@tok@sd\endcsname{\let\PY@it=\textit\def\PY@tc##1{\textcolor[rgb]{0.73,0.13,0.13}{##1}}}
\expandafter\def\csname PY@tok@si\endcsname{\let\PY@bf=\textbf\def\PY@tc##1{\textcolor[rgb]{0.73,0.40,0.53}{##1}}}
\expandafter\def\csname PY@tok@se\endcsname{\let\PY@bf=\textbf\def\PY@tc##1{\textcolor[rgb]{0.73,0.40,0.13}{##1}}}
\expandafter\def\csname PY@tok@sr\endcsname{\def\PY@tc##1{\textcolor[rgb]{0.73,0.40,0.53}{##1}}}
\expandafter\def\csname PY@tok@ss\endcsname{\def\PY@tc##1{\textcolor[rgb]{0.10,0.09,0.49}{##1}}}
\expandafter\def\csname PY@tok@sx\endcsname{\def\PY@tc##1{\textcolor[rgb]{0.00,0.50,0.00}{##1}}}
\expandafter\def\csname PY@tok@m\endcsname{\def\PY@tc##1{\textcolor[rgb]{0.40,0.40,0.40}{##1}}}
\expandafter\def\csname PY@tok@gh\endcsname{\let\PY@bf=\textbf\def\PY@tc##1{\textcolor[rgb]{0.00,0.00,0.50}{##1}}}
\expandafter\def\csname PY@tok@gu\endcsname{\let\PY@bf=\textbf\def\PY@tc##1{\textcolor[rgb]{0.50,0.00,0.50}{##1}}}
\expandafter\def\csname PY@tok@gd\endcsname{\def\PY@tc##1{\textcolor[rgb]{0.63,0.00,0.00}{##1}}}
\expandafter\def\csname PY@tok@gi\endcsname{\def\PY@tc##1{\textcolor[rgb]{0.00,0.63,0.00}{##1}}}
\expandafter\def\csname PY@tok@gr\endcsname{\def\PY@tc##1{\textcolor[rgb]{1.00,0.00,0.00}{##1}}}
\expandafter\def\csname PY@tok@ge\endcsname{\let\PY@it=\textit}
\expandafter\def\csname PY@tok@gs\endcsname{\let\PY@bf=\textbf}
\expandafter\def\csname PY@tok@gp\endcsname{\let\PY@bf=\textbf\def\PY@tc##1{\textcolor[rgb]{0.00,0.00,0.50}{##1}}}
\expandafter\def\csname PY@tok@go\endcsname{\def\PY@tc##1{\textcolor[rgb]{0.53,0.53,0.53}{##1}}}
\expandafter\def\csname PY@tok@gt\endcsname{\def\PY@tc##1{\textcolor[rgb]{0.00,0.27,0.87}{##1}}}
\expandafter\def\csname PY@tok@err\endcsname{\def\PY@bc##1{\setlength{\fboxsep}{0pt}\fcolorbox[rgb]{1.00,0.00,0.00}{1,1,1}{\strut ##1}}}
\expandafter\def\csname PY@tok@kc\endcsname{\let\PY@bf=\textbf\def\PY@tc##1{\textcolor[rgb]{0.00,0.50,0.00}{##1}}}
\expandafter\def\csname PY@tok@kd\endcsname{\let\PY@bf=\textbf\def\PY@tc##1{\textcolor[rgb]{0.00,0.50,0.00}{##1}}}
\expandafter\def\csname PY@tok@kn\endcsname{\let\PY@bf=\textbf\def\PY@tc##1{\textcolor[rgb]{0.00,0.50,0.00}{##1}}}
\expandafter\def\csname PY@tok@kr\endcsname{\let\PY@bf=\textbf\def\PY@tc##1{\textcolor[rgb]{0.00,0.50,0.00}{##1}}}
\expandafter\def\csname PY@tok@bp\endcsname{\def\PY@tc##1{\textcolor[rgb]{0.00,0.50,0.00}{##1}}}
\expandafter\def\csname PY@tok@fm\endcsname{\def\PY@tc##1{\textcolor[rgb]{0.00,0.00,1.00}{##1}}}
\expandafter\def\csname PY@tok@vc\endcsname{\def\PY@tc##1{\textcolor[rgb]{0.10,0.09,0.49}{##1}}}
\expandafter\def\csname PY@tok@vg\endcsname{\def\PY@tc##1{\textcolor[rgb]{0.10,0.09,0.49}{##1}}}
\expandafter\def\csname PY@tok@vi\endcsname{\def\PY@tc##1{\textcolor[rgb]{0.10,0.09,0.49}{##1}}}
\expandafter\def\csname PY@tok@vm\endcsname{\def\PY@tc##1{\textcolor[rgb]{0.10,0.09,0.49}{##1}}}
\expandafter\def\csname PY@tok@sa\endcsname{\def\PY@tc##1{\textcolor[rgb]{0.73,0.13,0.13}{##1}}}
\expandafter\def\csname PY@tok@sb\endcsname{\def\PY@tc##1{\textcolor[rgb]{0.73,0.13,0.13}{##1}}}
\expandafter\def\csname PY@tok@sc\endcsname{\def\PY@tc##1{\textcolor[rgb]{0.73,0.13,0.13}{##1}}}
\expandafter\def\csname PY@tok@dl\endcsname{\def\PY@tc##1{\textcolor[rgb]{0.73,0.13,0.13}{##1}}}
\expandafter\def\csname PY@tok@s2\endcsname{\def\PY@tc##1{\textcolor[rgb]{0.73,0.13,0.13}{##1}}}
\expandafter\def\csname PY@tok@sh\endcsname{\def\PY@tc##1{\textcolor[rgb]{0.73,0.13,0.13}{##1}}}
\expandafter\def\csname PY@tok@s1\endcsname{\def\PY@tc##1{\textcolor[rgb]{0.73,0.13,0.13}{##1}}}
\expandafter\def\csname PY@tok@mb\endcsname{\def\PY@tc##1{\textcolor[rgb]{0.40,0.40,0.40}{##1}}}
\expandafter\def\csname PY@tok@mf\endcsname{\def\PY@tc##1{\textcolor[rgb]{0.40,0.40,0.40}{##1}}}
\expandafter\def\csname PY@tok@mh\endcsname{\def\PY@tc##1{\textcolor[rgb]{0.40,0.40,0.40}{##1}}}
\expandafter\def\csname PY@tok@mi\endcsname{\def\PY@tc##1{\textcolor[rgb]{0.40,0.40,0.40}{##1}}}
\expandafter\def\csname PY@tok@il\endcsname{\def\PY@tc##1{\textcolor[rgb]{0.40,0.40,0.40}{##1}}}
\expandafter\def\csname PY@tok@mo\endcsname{\def\PY@tc##1{\textcolor[rgb]{0.40,0.40,0.40}{##1}}}
\expandafter\def\csname PY@tok@ch\endcsname{\let\PY@it=\textit\def\PY@tc##1{\textcolor[rgb]{0.25,0.50,0.50}{##1}}}
\expandafter\def\csname PY@tok@cm\endcsname{\let\PY@it=\textit\def\PY@tc##1{\textcolor[rgb]{0.25,0.50,0.50}{##1}}}
\expandafter\def\csname PY@tok@cpf\endcsname{\let\PY@it=\textit\def\PY@tc##1{\textcolor[rgb]{0.25,0.50,0.50}{##1}}}
\expandafter\def\csname PY@tok@c1\endcsname{\let\PY@it=\textit\def\PY@tc##1{\textcolor[rgb]{0.25,0.50,0.50}{##1}}}
\expandafter\def\csname PY@tok@cs\endcsname{\let\PY@it=\textit\def\PY@tc##1{\textcolor[rgb]{0.25,0.50,0.50}{##1}}}

\def\PYZbs{\char`\\}
\def\PYZus{\char`\_}
\def\PYZob{\char`\{}
\def\PYZcb{\char`\}}
\def\PYZca{\char`\^}
\def\PYZam{\char`\&}
\def\PYZlt{\char`\<}
\def\PYZgt{\char`\>}
\def\PYZsh{\char`\#}
\def\PYZpc{\char`\%}
\def\PYZdl{\char`\$}
\def\PYZhy{\char`\-}
\def\PYZsq{\char`\'}
\def\PYZdq{\char`\"}
\def\PYZti{\char`\~}
% for compatibility with earlier versions
\def\PYZat{@}
\def\PYZlb{[}
\def\PYZrb{]}
\makeatother


    % Exact colors from NB
    \definecolor{incolor}{rgb}{0.0, 0.0, 0.5}
    \definecolor{outcolor}{rgb}{0.545, 0.0, 0.0}

    % Don't number sections
    \renewcommand{\thesection}{\hspace*{-0.5em}}
    \renewcommand{\thesubsection}{\hspace*{-0.5em}}



    
    % Prevent overflowing lines due to hard-to-break entities
    \sloppy 
    % Setup hyperref package
    \hypersetup{
      breaklinks=true,  % so long urls are correctly broken across lines
      colorlinks=true,
      urlcolor=urlcolor,
      linkcolor=linkcolor,
      citecolor=citecolor,
      }
    % Slightly bigger margins than the latex defaults
    
    \geometry{verbose,tmargin=1in,bmargin=1in,lmargin=1in,rmargin=1in}
    
    

    \begin{document}
    
    
    
    
    

    
    \hypertarget{session-3-handout-solutions-only}{%
\section{Session 3 Handout (Solutions
Only)}\label{session-3-handout-solutions-only}}

    \textbf{Q1:} (Modification of case 2 from last session) Write a function
named \texttt{orderQuantity} that takes two input arguments,
\texttt{inventory} and \texttt{basestock}. If \texttt{inventory} is at
least equal to \texttt{basestock}, then return 0. Otherwise, return the
difference between \texttt{basestock} and \texttt{inventory}. Set the
default value for \texttt{inventory} to be 0 and for \texttt{basestock}
to be 100. Include an appropriate docstring to explain what the function
does.

	
\begin{Verbatim}[commandchars=\\\{\}]
{\color{incolor}[{\color{incolor}2}]:} \PY{k}{def} \PY{n+nf}{orderQuantity}\PY{p}{(}\PY{n}{inventory}\PY{o}{=}\PY{l+m+mi}{0}\PY{p}{,}\PY{n}{basestock}\PY{o}{=}\PY{l+m+mi}{100}\PY{p}{)}\PY{p}{:}
         \PY{l+s+sd}{\PYZsq{}\PYZsq{}\PYZsq{} Calculates order quantity given inventory level and basestock level\PYZsq{}\PYZsq{}\PYZsq{}}
         \PY{k}{if} \PY{n}{inventory}\PY{o}{\PYZgt{}}\PY{o}{=}\PY{n}{basestock}\PY{p}{:}
             \PY{k}{return} \PY{l+m+mi}{0}
         \PY{k}{else}\PY{p}{:}
             \PY{k}{return} \PY{n}{basestock}\PY{o}{\PYZhy{}}\PY{n}{inventory}
\end{Verbatim}
	
\begin{Verbatim}[commandchars=\\\{\}]
{\color{incolor}[{\color{incolor}3}]:} \PY{c+c1}{\PYZsh{} Code to test your function}
     \PY{n}{help}\PY{p}{(}\PY{n}{orderQuantity}\PY{p}{)}
     \PY{n+nb}{print}\PY{p}{(}\PY{n}{orderQuantity}\PY{p}{(}\PY{p}{)}\PY{p}{)}
     \PY{n+nb}{print}\PY{p}{(}\PY{n}{orderQuantity}\PY{p}{(}\PY{l+m+mi}{25}\PY{p}{)}\PY{p}{)}
     \PY{n+nb}{print}\PY{p}{(}\PY{n}{orderQuantity}\PY{p}{(}\PY{l+m+mi}{51}\PY{p}{,}\PY{l+m+mi}{50}\PY{p}{)}\PY{p}{)}
     \PY{n+nb}{print}\PY{p}{(}\PY{n}{orderQuantity}\PY{p}{(}\PY{n}{basestock}\PY{o}{=}\PY{l+m+mi}{200}\PY{p}{)}\PY{p}{)}
     \PY{n+nb}{print}\PY{p}{(}\PY{n}{orderQuantity}\PY{p}{(}\PY{n}{inventory}\PY{o}{=}\PY{l+m+mi}{80}\PY{p}{)}\PY{p}{)}
\end{Verbatim}
\begin{Verbatim}[commandchars=\\\{\}]
Help on function orderQuantity in module \_\_main\_\_:

orderQuantity(inventory=0, basestock=100)
    Calculates order quantity given inventory level and basestock level

100
75
0
200
20

\end{Verbatim}

    \textbf{Q2:} Walk through the code to explain each line of the above
output.

    \textbf{Q3:} Run the above line and out of the items in all lowercase,
choose five that look interesting to you, and use \texttt{type} and
\texttt{help} and trial and error to find out what each of these
built-in objects are and what you can do with them. Explain to your
neighbor.

	
\begin{Verbatim}[commandchars=\\\{\}]
{\color{incolor}[{\color{incolor}9}]:} \PY{n}{help}\PY{p}{(}\PY{n+nb}{abs}\PY{p}{)}
\end{Verbatim}
\begin{Verbatim}[commandchars=\\\{\}]
Help on built-in function abs in module builtins:

abs(x, /)
    Return the absolute value of the argument.


\end{Verbatim}

	
\begin{Verbatim}[commandchars=\\\{\}]
{\color{incolor}[{\color{incolor}10}]:} \PY{n}{help}\PY{p}{(}\PY{n+nb}{max}\PY{p}{)}
\end{Verbatim}
\begin{Verbatim}[commandchars=\\\{\}]
Help on built-in function max in module builtins:

max({\ldots})
    max(iterable, *[, default=obj, key=func]) -> value
    max(arg1, arg2, *args, *[, key=func]) -> value
    
    With a single iterable argument, return its biggest item. The
    default keyword-only argument specifies an object to return if
    the provided iterable is empty.
    With two or more arguments, return the largest argument.


\end{Verbatim}

	
\begin{Verbatim}[commandchars=\\\{\}]
{\color{incolor}[{\color{incolor}11}]:} \PY{n}{help}\PY{p}{(}\PY{n+nb}{min}\PY{p}{)}
\end{Verbatim}
\begin{Verbatim}[commandchars=\\\{\}]
Help on built-in function min in module builtins:

min({\ldots})
    min(iterable, *[, default=obj, key=func]) -> value
    min(arg1, arg2, *args, *[, key=func]) -> value
    
    With a single iterable argument, return its smallest item. The
    default keyword-only argument specifies an object to return if
    the provided iterable is empty.
    With two or more arguments, return the smallest argument.


\end{Verbatim}

	
\begin{Verbatim}[commandchars=\\\{\}]
{\color{incolor}[{\color{incolor}12}]:} \PY{n}{help}\PY{p}{(}\PY{n+nb}{sum}\PY{p}{)}
\end{Verbatim}
\begin{Verbatim}[commandchars=\\\{\}]
Help on built-in function sum in module builtins:

sum(iterable, start=0, /)
    Return the sum of a 'start' value (default: 0) plus an iterable of numbers
    
    When the iterable is empty, return the start value.
    This function is intended specifically for use with numeric values and may
    reject non-numeric types.


\end{Verbatim}

    \textbf{Q4:} Import the \texttt{math} module and print the list of
variables and functions within this module using \texttt{dir}. Choose
five functions from this list and use \texttt{help} and trial and error
to figure out how to use them. Explain to your neighbor.

	
\begin{Verbatim}[commandchars=\\\{\}]
{\color{incolor}[{\color{incolor}13}]:} \PY{k+kn}{import} \PY{n+nn}{math}
      \PY{n+nb}{print}\PY{p}{(}\PY{n+nb}{dir}\PY{p}{(}\PY{n}{math}\PY{p}{)}\PY{p}{)}
\end{Verbatim}
\begin{Verbatim}[commandchars=\\\{\}]
['\_\_doc\_\_', '\_\_loader\_\_', '\_\_name\_\_', '\_\_package\_\_', '\_\_spec\_\_', 'acos', 'acosh', 'asin', 'asinh', 'atan', 'atan2', 'atanh', 'ceil', 'copysign', 'cos', 'cosh', 'degrees', 'e', 'erf', 'erfc', 'exp', 'expm1', 'fabs', 'factorial', 'floor', 'fmod', 'frexp', 'fsum', 'gamma', 'gcd', 'hypot', 'inf', 'isclose', 'isfinite', 'isinf', 'isnan', 'ldexp', 'lgamma', 'log', 'log10', 'log1p', 'log2', 'modf', 'nan', 'pi', 'pow', 'radians', 'sin', 'sinh', 'sqrt', 'tan', 'tanh', 'tau', 'trunc']

\end{Verbatim}

	
\begin{Verbatim}[commandchars=\\\{\}]
{\color{incolor}[{\color{incolor}14}]:} \PY{n}{help}\PY{p}{(}\PY{n}{math}\PY{o}{.}\PY{n}{cos}\PY{p}{)}
\end{Verbatim}
\begin{Verbatim}[commandchars=\\\{\}]
Help on built-in function cos in module math:

cos({\ldots})
    cos(x)
    
    Return the cosine of x (measured in radians).


\end{Verbatim}

	
\begin{Verbatim}[commandchars=\\\{\}]
{\color{incolor}[{\color{incolor}15}]:} \PY{n}{help}\PY{p}{(}\PY{n}{math}\PY{o}{.}\PY{n}{log}\PY{p}{)}
\end{Verbatim}
\begin{Verbatim}[commandchars=\\\{\}]
Help on built-in function log in module math:

log({\ldots})
    log(x[, base])
    
    Return the logarithm of x to the given base.
    If the base not specified, returns the natural logarithm (base e) of x.


\end{Verbatim}

    \textbf{Q5:} Use \texttt{dir} on the string object \texttt{"Hi"}. Choose
five functions from this list and use \texttt{help} and trial and error
to figure out how to use these functions built in to every string
object. Explain to your neighbor.

	
\begin{Verbatim}[commandchars=\\\{\}]
{\color{incolor}[{\color{incolor}16}]:} \PY{n+nb}{print}\PY{p}{(}\PY{n+nb}{dir}\PY{p}{(}\PY{l+s+s1}{\PYZsq{}}\PY{l+s+s1}{Hi}\PY{l+s+s1}{\PYZsq{}}\PY{p}{)}\PY{p}{)}
\end{Verbatim}
\begin{Verbatim}[commandchars=\\\{\}]
['\_\_add\_\_', '\_\_class\_\_', '\_\_contains\_\_', '\_\_delattr\_\_', '\_\_dir\_\_', '\_\_doc\_\_', '\_\_eq\_\_', '\_\_format\_\_', '\_\_ge\_\_', '\_\_getattribute\_\_', '\_\_getitem\_\_', '\_\_getnewargs\_\_', '\_\_gt\_\_', '\_\_hash\_\_', '\_\_init\_\_', '\_\_init\_subclass\_\_', '\_\_iter\_\_', '\_\_le\_\_', '\_\_len\_\_', '\_\_lt\_\_', '\_\_mod\_\_', '\_\_mul\_\_', '\_\_ne\_\_', '\_\_new\_\_', '\_\_reduce\_\_', '\_\_reduce\_ex\_\_', '\_\_repr\_\_', '\_\_rmod\_\_', '\_\_rmul\_\_', '\_\_setattr\_\_', '\_\_sizeof\_\_', '\_\_str\_\_', '\_\_subclasshook\_\_', 'capitalize', 'casefold', 'center', 'count', 'encode', 'endswith', 'expandtabs', 'find', 'format', 'format\_map', 'index', 'isalnum', 'isalpha', 'isdecimal', 'isdigit', 'isidentifier', 'islower', 'isnumeric', 'isprintable', 'isspace', 'istitle', 'isupper', 'join', 'ljust', 'lower', 'lstrip', 'maketrans', 'partition', 'replace', 'rfind', 'rindex', 'rjust', 'rpartition', 'rsplit', 'rstrip', 'split', 'splitlines', 'startswith', 'strip', 'swapcase', 'title', 'translate', 'upper', 'zfill']

\end{Verbatim}

	
\begin{Verbatim}[commandchars=\\\{\}]
{\color{incolor}[{\color{incolor}17}]:} \PY{n}{help}\PY{p}{(}\PY{l+s+s1}{\PYZsq{}}\PY{l+s+s1}{Hi}\PY{l+s+s1}{\PYZsq{}}\PY{o}{.}\PY{n}{lower}\PY{p}{)}
\end{Verbatim}
\begin{Verbatim}[commandchars=\\\{\}]
Help on built-in function lower:

lower({\ldots}) method of builtins.str instance
    S.lower() -> str
    
    Return a copy of the string S converted to lowercase.


\end{Verbatim}

	
\begin{Verbatim}[commandchars=\\\{\}]
{\color{incolor}[{\color{incolor}18}]:} \PY{n}{help}\PY{p}{(}\PY{n+nb}{str}\PY{o}{.}\PY{n}{lower}\PY{p}{)}
\end{Verbatim}
\begin{Verbatim}[commandchars=\\\{\}]
Help on method\_descriptor:

lower({\ldots})
    S.lower() -> str
    
    Return a copy of the string S converted to lowercase.


\end{Verbatim}

	
\begin{Verbatim}[commandchars=\\\{\}]
{\color{incolor}[{\color{incolor}19}]:} \PY{n}{help}\PY{p}{(}\PY{n+nb}{str}\PY{o}{.}\PY{n}{find}\PY{p}{)}
\end{Verbatim}
\begin{Verbatim}[commandchars=\\\{\}]
Help on method\_descriptor:

find({\ldots})
    S.find(sub[, start[, end]]) -> int
    
    Return the lowest index in S where substring sub is found,
    such that sub is contained within S[start:end].  Optional
    arguments start and end are interpreted as in slice notation.
    
    Return -1 on failure.


\end{Verbatim}

    \hypertarget{case-6a.-mortgage-calculator-i}{%
\subsubsection{Case 6a. Mortgage Calculator
I}\label{case-6a.-mortgage-calculator-i}}

Write a function \texttt{numberMonths} in module \texttt{session3} that
calculates how many months it would take to pay off a mortgage given the
monthly payment. The function has four input arguments: \texttt{total},
\texttt{monthly}, \texttt{interest}, and \texttt{downpay}. Let the
default values for \texttt{interest} be 0.0425 and for \texttt{downpay}
be 0. Label the four arguments \(T\), \(M\), \(I\), \(D\) respectively.
The number of months needed \(N\) is given by the formula

\[ N = ceil\left( \frac{-\log(1-\frac{i(T-D)}{M})}{\log(1+i)} \right), \]
where \(i=I/12\) is the monthly interest rate and \(ceil\) is the
\texttt{math.ceil} function. (Note, after modifying the
\texttt{session3.py}, you will have to restart the kernel using the
toolbar above to reload the latest version.)

	
\begin{Verbatim}[commandchars=\\\{\}]
{\color{incolor}[{\color{incolor}20}]:} \PY{k+kn}{import} \PY{n+nn}{session3} \PY{k}{as} \PY{n+nn}{s3}
      \PY{n}{s3}\PY{o}{.}\PY{n}{numberMonths}\PY{p}{(}\PY{l+m+mi}{500000}\PY{p}{,}\PY{l+m+mi}{4000}\PY{p}{)}\PY{o}{/}\PY{l+m+mi}{12}
\end{Verbatim}
	

    
    
\begin{verbatim}
13.833333333333334
\end{verbatim}

    

	
\begin{Verbatim}[commandchars=\\\{\}]
{\color{incolor}[{\color{incolor}21}]:} \PY{n}{s3}\PY{o}{.}\PY{n}{numberMonths}\PY{p}{(}\PY{l+m+mi}{500000}\PY{p}{,}\PY{l+m+mi}{4000}\PY{p}{,}\PY{n}{interest}\PY{o}{=}\PY{l+m+mf}{0.05}\PY{p}{)}\PY{o}{/}\PY{l+m+mi}{12}
\end{Verbatim}
	

    
    
\begin{verbatim}
14.75
\end{verbatim}

    

    \hypertarget{case-6b.-mortgage-calculator-ii}{%
\subsubsection{Case 6b. Mortgage Calculator
II}\label{case-6b.-mortgage-calculator-ii}}

Write a function \texttt{monthlyPayment} in module \texttt{session3}
that calculates the monthly payment needed to pay off a mortgage in a
given number of months. The function has four input arguments:
\texttt{total}, \texttt{months}, \texttt{interest}, and
\texttt{downpay}. Let the default values for \texttt{interest} be 0.0425
and for \texttt{downpay} be 0. Label the four arguments \(T\), \(N\),
\(I\), \(D\) respectively. The monthly payment \(M\) is given by the
formula

\[M=\frac{(1+i)^N}{(1+i)^N-1}i(T-D),\] where \(i=I/12\) is the monthly
interest rate. Round the answer to two decimal places using the
\texttt{round} function.

	
\begin{Verbatim}[commandchars=\\\{\}]
{\color{incolor}[{\color{incolor}22}]:} \PY{n}{s3}\PY{o}{.}\PY{n}{monthlyPayment}\PY{p}{(}\PY{l+m+mi}{500000}\PY{p}{,}\PY{l+m+mi}{12}\PY{o}{*}\PY{l+m+mi}{30}\PY{p}{)}
\end{Verbatim}
	

    
    
\begin{verbatim}
2459.7
\end{verbatim}

    

	
\begin{Verbatim}[commandchars=\\\{\}]
{\color{incolor}[{\color{incolor}23}]:} \PY{n}{s3}\PY{o}{.}\PY{n}{monthlyPayment}\PY{p}{(}\PY{l+m+mi}{500000}\PY{p}{,}\PY{l+m+mi}{12}\PY{o}{*}\PY{l+m+mi}{30}\PY{p}{,}\PY{n}{interest}\PY{o}{=}\PY{l+m+mf}{0.05}\PY{p}{)}
\end{Verbatim}
	

    
    
\begin{verbatim}
2684.11
\end{verbatim}

    

    The two functions are below.

	
\begin{Verbatim}[commandchars=\\\{\}]
{\color{incolor}[{\color{incolor} }]:} \PY{k+kn}{import} \PY{n+nn}{math}
     \PY{k}{def} \PY{n+nf}{numberMonths}\PY{p}{(}\PY{n}{total}\PY{p}{,}\PY{n}{monthly}\PY{p}{,}\PY{n}{interest}\PY{o}{=}\PY{l+m+mf}{0.0425}\PY{p}{,}\PY{n}{downpay}\PY{o}{=}\PY{l+m+mi}{0}\PY{p}{)}\PY{p}{:}
         \PY{n}{i}\PY{o}{=}\PY{n}{interest}\PY{o}{/}\PY{l+m+mi}{12}
         \PY{k}{return} \PY{n}{math}\PY{o}{.}\PY{n}{ceil}\PY{p}{(}\PY{o}{\PYZhy{}}\PY{n}{math}\PY{o}{.}\PY{n}{log}\PY{p}{(}\PY{l+m+mi}{1}\PY{o}{\PYZhy{}}\PY{n}{i}\PY{o}{*}\PY{p}{(}\PY{n}{total}\PY{o}{\PYZhy{}}\PY{n}{downpay}\PY{p}{)}\PY{o}{/}\PY{n}{monthly}\PY{p}{)}\PY{o}{/}\PY{n}{math}\PY{o}{.}\PY{n}{log}\PY{p}{(}\PY{l+m+mi}{1}\PY{o}{+}\PY{n}{i}\PY{p}{)}\PY{p}{)}
     
     \PY{k}{def} \PY{n+nf}{monthlyPayment}\PY{p}{(}\PY{n}{total}\PY{p}{,}\PY{n}{months}\PY{p}{,}\PY{n}{interest}\PY{o}{=}\PY{l+m+mf}{0.0425}\PY{p}{,}\PY{n}{downpay}\PY{o}{=}\PY{l+m+mi}{0}\PY{p}{)}\PY{p}{:}
         \PY{n}{i}\PY{o}{=}\PY{n}{interest}\PY{o}{/}\PY{l+m+mi}{12}
         \PY{k}{return} \PY{n+nb}{round}\PY{p}{(}\PY{p}{(}\PY{l+m+mi}{1}\PY{o}{+}\PY{n}{i}\PY{p}{)}\PY{o}{*}\PY{o}{*}\PY{n}{months}\PY{o}{*}\PY{p}{(}\PY{n}{total}\PY{o}{\PYZhy{}}\PY{n}{downpay}\PY{p}{)}\PY{o}{*}\PY{n}{i}\PY{o}{/}\PY{p}{(}\PY{p}{(}\PY{l+m+mi}{1}\PY{o}{+}\PY{n}{i}\PY{p}{)}\PY{o}{*}\PY{o}{*}\PY{n}{months}\PY{o}{\PYZhy{}}\PY{l+m+mi}{1}\PY{p}{)}\PY{p}{,}\PY{l+m+mi}{2}\PY{p}{)}
\end{Verbatim}
    Note that it is easier to debug if you disected the formula into small
chuncks, as below

	
\begin{Verbatim}[commandchars=\\\{\}]
{\color{incolor}[{\color{incolor} }]:} \PY{k+kn}{import} \PY{n+nn}{math}
     \PY{k}{def} \PY{n+nf}{numberMonths}\PY{p}{(}\PY{n}{total}\PY{p}{,}\PY{n}{monthly}\PY{p}{,}\PY{n}{interest}\PY{o}{=}\PY{l+m+mf}{0.0425}\PY{p}{,}\PY{n}{downpay}\PY{o}{=}\PY{l+m+mi}{0}\PY{p}{)}\PY{p}{:}
         \PY{n}{i}\PY{o}{=}\PY{n}{interest}\PY{o}{/}\PY{l+m+mi}{12}
         \PY{n}{A}\PY{o}{=}\PY{n}{i}\PY{o}{*}\PY{p}{(}\PY{n}{total}\PY{o}{\PYZhy{}}\PY{n}{downpay}\PY{p}{)}\PY{o}{/}\PY{n}{monthly}
         \PY{n}{top}\PY{o}{=}\PY{o}{\PYZhy{}}\PY{n}{math}\PY{o}{.}\PY{n}{log}\PY{p}{(}\PY{l+m+mi}{1}\PY{o}{\PYZhy{}}\PY{n}{A}\PY{p}{)}
         \PY{n}{bottom}\PY{o}{=}\PY{n}{math}\PY{o}{.}\PY{n}{log}\PY{p}{(}\PY{l+m+mi}{1}\PY{o}{+}\PY{n}{i}\PY{p}{)}
         \PY{k}{return} \PY{n}{math}\PY{o}{.}\PY{n}{ceil}\PY{p}{(}\PY{n}{top}\PY{o}{/}\PY{n}{bottom}\PY{p}{)}
     
     \PY{k}{def} \PY{n+nf}{monthlyPayment}\PY{p}{(}\PY{n}{total}\PY{p}{,}\PY{n}{months}\PY{p}{,}\PY{n}{interest}\PY{o}{=}\PY{l+m+mf}{0.0425}\PY{p}{,}\PY{n}{downpay}\PY{o}{=}\PY{l+m+mi}{0}\PY{p}{)}\PY{p}{:}
         \PY{n}{i}\PY{o}{=}\PY{n}{interest}\PY{o}{/}\PY{l+m+mi}{12}
         \PY{n}{top}\PY{o}{=}\PY{p}{(}\PY{l+m+mi}{1}\PY{o}{+}\PY{n}{i}\PY{p}{)}\PY{o}{*}\PY{o}{*}\PY{n}{months}\PY{o}{*}\PY{n}{i}\PY{o}{*}\PY{p}{(}\PY{n}{total}\PY{o}{\PYZhy{}}\PY{n}{downpay}\PY{p}{)}
         \PY{n}{bottom}\PY{o}{=}\PY{p}{(}\PY{l+m+mi}{1}\PY{o}{+}\PY{n}{i}\PY{p}{)}\PY{o}{*}\PY{o}{*}\PY{n}{months}\PY{o}{\PYZhy{}}\PY{l+m+mi}{1}
         \PY{k}{return} \PY{n+nb}{round}\PY{p}{(}\PY{n}{top}\PY{o}{/}\PY{n}{bottom}\PY{p}{,}\PY{l+m+mi}{2}\PY{p}{)}
\end{Verbatim}

    % Add a bibliography block to the postdoc
    
    
    
    \end{document}
